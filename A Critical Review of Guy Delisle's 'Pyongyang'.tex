\documentclass{article}
\usepackage[utf8]{inputenc}

\title{A CRITICAL REVIEW OF GUY DELISLE’S \textit{PYONGYANG}:
THE TRAP OF THE WESTERNER’S IGNORANCE}
\author{Christina Nguyen }
\date{January 23 2021}


\begin{document}

\maketitle


“There is a vision of the Orient that I [the Westerner] have [.] It is a vision that has become my life.”   
\newline - from \textit{M.  Butterfly} (Hwang 1988, 91).

\section{Introduction}

\qquad \textit{Pyongyang: a Journey in North Korea} is the story of a Francophone Canadian animator named Guy Delisle who spends two months in North Korea for his work. His stories are humorous because of his bumbling experiences as a Westerner in North Korea. Additionally, the over-simplistic view of his observations allows for some superiority bias and incongruous humour. Unfortunately, this Western bias also leads to a tale full of inaccuracies and possibly misinformation; thus, it is necessary to provide several critical analyses into this light-hearted testimony. \\

 \newline \qquad Like in Giacomo Puccini’s \textit{Madama Butterfly}, Delisle’s \textit{Pyongyang} presents to the Westerner precisely what he expects to see in Asia, due to his ignorance of North Korea and its culture. Let me preface by saying that I do not argue that North Korea is elegant in every way; and I am not arguing for the political regime of North Korea and all its violence. However, one must recognize that Delisle is presenting a very \textit{selective} (i.e. precise but not accurate) view of his journey in North Korea. For the most part, the documentation of his experience is extremely precise though; so, credit must be given there. Nonetheless, Delisle is hand-picking the most gossip-worthy parts of his experience, which he knows will make for an exciting and “sellable” novel which may bear only a superficial semblance to the truth. Like Puccini’s \textit{Madama Butterfly}, it is “a [discourse] that informs [the Westerner’s] material realities within which the Orient becomes a European product for Western consumption” (Nieves 2014, 21). \\

\section{Part A: Summary of the Plot}

“Our country itself was cursed, bastardized, partitioned into North and South and if it could be said of us that we chose division and death in our uncivil war, that was also only partially true. We had not chosen to be debased by the [West…] and to be turned over to the great powers of capitalism and communism for a further bisection, then given roles as the clashing armies of a Cold War chess match played in air-conditioned rooms by white men wearing suits and ties.” 
\\ - from \textit{The Sympathizer} (Nguyen 2015, 108). \\

\newline \qquad Firstly, we must summarize the contents of the graphic novel. What did Delisle do, when did he do it, and how did he record these events? What tone of language does Delisle use? How does he view the people of North Korea? Then, using these observations, we can analyse his biases in the second and third part of this critical review. These analyses will consider Delisle’s literary style and content, accordingly. \\


\newline \qquad Guy Delisle passes through North Korea’s port of entry, where he is questioned about his motives in visiting. Receiving a guide, we see his dreary trip from the port of entry towards his hotel. A long road covered in mist and fog is shown. Delisle interjects with the tips he has been given prior to his journey, notes like “[N]o lights at night” (Delisle 2015, 4). The guide and he make their way to a large statue of Leader Kim Il-Sung where he places some flowers at its feet.
\\

\newline Arriving at his hotel, Delisle immediately notes that “it’s got those good old standard rooms – cold and impersonal, just like they like them in Asia” (Delisle 2015, 9). A journey to his studios ensues. The elevator to the studio is slow, and Delisle asks knowingly, “[Do you] think it works?” (Delisle 2015, 1). The guide and he climb the stairs. Once in the office, he meets his old friend Sandrine, who says, “So, first Saigon and now Pyongyang!” (Delisle 2015, 12). After some light work, they go looking for food and manage to escape their guards for a brief while. Next comes a full-page replication of a poster Delisle saw in the city, saying, “Forging ahead into the 21st century!” (Delisle 2015, 17). \\

\newline \qquad Arriving at the hotel restaurant, Delisle makes some brief comments on the less-than-ideal state of the furniture and food. Sandrine points out some Chinese people walking backwards as an exercise, which further exemplifies the strangeness of this far-flung region of Asia for the reader (who is probably thinking, \textit{Huh? The} Chinese \textit{walking} backwards \textit{in North Korea? How did this hodgepodge of ideas get thrown together?}). Delisle then launches into another North Korea-is-crazy session; he describes the ‘too-clean’ streets, the production assistant who sings strange music, the pictures of the Kims hanging on the walls. At this point, I would like to interject and remind Mr. Delisle of the abundance of images of the Queen hanging around governmental buildings in Canada and other Commonwealth nations. Our royal (not national, but \textit{royal}) anthem is still ‘God save the Queen’, although we are mostly independent from the British monarchy (Government of Canada 2018, par. 8). \\

\newline \qquad Continuing on, Delisle also visits the Pyongyang metro (subway system), the Diplomatic Club (where he spies a few other foreigners and enjoys some time to relax), sees the Juche Tower, and a smattering of other museums. In all of these locations, he observes some outlandish customs; for example, after visiting a calm, serene park, he is surprised to discover the name of one of the Kims carved into a cliff-face. \\

\qquad Delisle also records things like the lack of disabled people, the ‘cult of personality’ for the rulers of North Korea, the not-so-secret spying of his guide (who is more of a government informant), and the nearly-blasphemous Coca-Cola (which he drinks although he does not enjoy it, just to prove a point). At the end of the story, the reader comes to view the novel as a hodgepodge of shorter stories that really has no significant plot; nothing has really happened in the larger scheme of things. \textit{So what?}, the reader feels like asking. All that was presented were the strange little rules that North Korea imposes on its people; many of us probably knew these things already from other, more reliable, sources.\\

\section{Part B: Analysis of Style}

“Whatever people say about the General today, I can only testify that he was a sincere man who believed in everything he said, even if it was a lie, which makes him not so different to most.” \\
- from \textit{The Sympathizer} (Nguyen 2015, 12). \\

I. The title \\

\newline \qquad Okay, I hear you! But the title isn’t a stylistic choice; it’s part of the content, you say. However, the title is also the very first object you see on the cover of the graphic novel. Delisle does not call the country “The Democratic People’s Republic of Korea” (its official name); rather he calls it by the Western name of North Korea. The political implications here are endless.\\ 

 \qquad \textit{But he wrote it in French originally! How can you know that this was his intention, rather than the translator’s?} you ask. Let us take a look at page 173 from the French version of the graphic novel. In it, Delisle writes “Extrait 3: Encerclé par l’ennemi, un soldat nord-coréen, incendie sa maison plutôt que de capituler devant les capitalistes", or "Surrounded by the enemy, a North Korean soldier burns his house rather than capitulate to the capitalists” (Delisle 2015, 173). \\

\qquad Thus the French text calls the country “Nord Corée” (North Korea) as well, rather than “la République populaire démocratique de Corée.” In this case, we cede to the political stance that Guy Delisle has certainly taken. \\

II. On the colour scheme \\

\qquad The first factor that strikes when reading \textit{Pyongyang} is the black-and-white color scheme. This depiction is rare; most of the images we have of Communist Korea (or any communist nation) is rubicund. Red lines the streets of imagination when we think of the Leninist Soviet Union, Maoist China, or North Korea. Is Delisle trying to convey a subtle message here; is he implying that North Korea is devoid of emotion or passion? Or that North Korea is dissimilar to other far-left nations of our knowledge? \\

\qquad Upon looking at his other works, we notice that Delisle has also used the black-and-white scheme for \textit{Shenzhen: A Travelogue from China} (2000), \textit{Jerusalem: Chronicles from the Holy City} (2011), and \textit{Burma Chronicles} (2007). Jerusalem is a parliamentary democracy and not leftist, so there is no connection to other leftist states that Delisle is drawing here. Therefore, the most likely explanation for this color scheme is that it is simply his personal style. So, we will not read into this choice further. \\

\qquad But wait! There is also the minor factor of the title page to take into consideration. In the Drawn and Quarterly 2015 edition of the book, the title page is in colour: yellow and red adorns it. Red, of course, is the first colour that the human eyes see, so the colour choice is probably a marketing technique to sell Delisle’s books better (rather than a black-and-white cover). \\

III. Why did Delisle write his testimony as a graphic novel? \\

\qquad Guy Delisle is, as we previously mentioned, an animator; thus, it is not strange that he should draw out his experiences alongside his words. Again, he has done this for his other works on travelling, so the images may not be a particular commentary on North Korea. However, it does present an effective method of communication in helping the reader visualize what he saw. It is also widely known that foreigners are banned from taking photos in most parts of North Korea, so drawing may have been the only way to capture his journey there. \\

\qquad Further, the fonts and informal drawings makes the book more accessible to non-academic readers who are interested in the workings of North Korea. Academics would gravitate toward more credible papers rather than books like these, but Delisle’s comical style makes it easier for the general public to read.\\

\qquad Additionally, his \textit{bande-desinées} style is a mix between the Franco-Belgian styles of comic-dynamic style (exaggerated views of reality) and schematic style (\texit{ligne clair} style). For the most part, his drawings are a reductionist view of reality to clear lines; we see only the most important objects. However, we also see exaggerated features of his face, for example. (Justcomix 2011, par. 4).

\section{Part C: Analysis of Content Bias}

“It’s not what you look at that matters; it’s what you see.” \\ 
- from \textit{Walden}  (Thoreau 1854, page unknown). \\

There lie too many biases in Guy Delisle’s Pyongyang to count; we will separate them by category as follows: (I) the absurd situations, ignorance on the Westerner’s part (II) day-to-day life of North Koreans (III) North Korean and South Korean attitudes towards reunification. \\

I. The absurd situations, ignorance on the Westerner’s part \\

\qquad One particularly silly example is Delisle’s apprehension to removing his slippers at a museum he visits (Delisle 2015, 103). Removing shoes and donning slippers is a common custom is most of Central Asia and the Pacific Rim, particularly in Japan, for example. To remove one’s shoes and use the provided slippers is to show respect for the insitution’s cleanliness and to maintain general hygiene. For Delisle to observe this as strange makes it seem markedly North Korean and not normal anywhere else. He falls into the fallacy of \textit{ad hominem}, reducing the reader’s judgement of North Korea (assuming that this custom does not occur elsewhere) to one particular action that can be found in Japan and South Korea as well.\\

\qquad Also, Delisle notes with shock the name of one of the Kims scraped into a cliff face; this should not be a surprise as other religions in other countries do this too. In Cambodia, for example, it is not uncommon to see gods carved into rocks alongside the road in the forests. It is not purely a North Korean ‘thing’ as he might lead us to believe. \\

II. Day-to-day life of North Koreans and policies \\

\qquad Reading this testimony, I recognized that Guy Delisle fails to give insight to some key aspects of daily life for the average citizen, such as grocery shopping or eating dinner with family, for example. This is through no fault of his own (him being an outsider), but this point of view limits the true understanding of North Korea’s societal values and traditions. We (the readers) see only what is in front of us, further reinforcing the idea of “[the Westerner’s] material realities within which the Orient becomes a European product for Western consumption” (Nieves 2014, 21). \\

\qquad Also disturbing is the vignette of Delisle presenting his guides with a bottle of fine wine at the end of the bande-desinées. He says that he sees a “rare moment of joy” on their faces and savours the idea that he brought this about (Delisle 2015, 175). The hegemonic idea of the ‘white savior’ must die; Delisle basks in the (possibly) false idea that he has ‘saved’ these people from some horrible, joyless life by giving them a little alcohol. This concept feeds into the “false-paradigm model” – the idea that some poor, godforsaken country needs the help of privileged white experts to regain its strength. This idea must die. \\

III. North Korean and South Korean attitudes towards reunification \\

\qquad One last little detail that guts me is how Delisle deals with the two Korea’s attitudes towards reunification. Guy Delisle paints the image that South Koreans have some contempt for reunification with North Koreas because it would cost too much, yet some South Koreans do say that it would be a great economic investment because of the large amount of natural resources lying North Korea’s ground (Asian Boss 2018). \\

\qquad It is also imperative to add that both Koreas express an interest in reunification and refer to the other Korea as ‘us’ (Al Jazeera 2018, par. 1). \\

\section{Part D: Conclusion} \\

“We are paid for our suspicions by finding what we suspected.” \\
- from \textit{A Week on the Concord and Merrimack Rivers} (Thoreau 1849, page unknown. \\

\qquad It is regrettable that Guy Delisle consciously chooses to block out many of North Korea’s less-known aspects to paint a more shocking and “sellable” portrait of North Korea. By withholding many key points of information, he leads us to an incomplete view of North Korean society. Delisle is, unfortunately, ethnocentric to a fault, projecting his expectations of the autocratic state into the novel. These are the shortcomings of his tale. However, we must acknowledge that Delisle’s graphic novel is a good starting point for those unfamiliar with the Westerner’s journey in North Korea’s autocratic system, and is a perfectly light read that is enjoyable anytime. \\


\newpage

\section{Bibliography} 

\\ \hspace*{0.3cm} 2018. \textit{A South and North Korean Meet For The First Time.} Produced by  
\newline \hspace*{2cm} Asian Boss. Accessed November 16, 2018. https://www.youtube.com/watch?v=Djv \\
\hspace*{2cm} MW8A4ktY. \\

Delisle, Guy. 2015. \textit{Pyongyang: A Journey in North Korea.} Drawn and \hspace*{2cm} Quarterly.\\

Hwang, David Henry. 1988. \textit {M. Butterfly.} https://macaulay.cuny.edu/eportfolios/gillespie17 \\
\hspace*{2cm} /files/2017/10/Hwang-David-Henry-M-Butterfly.pdf. \\

2011. \textit{Ligne Claire.} January 11. Accessed November 16, 2018. http://justcomix.com/2011/01/ \\ \hspace*{2cm} ligne-claire. \\

Nguyen, Viet Thanh. 2015. \textit{The Sympathizer.} New York: Grover Press. \\

\newline Nieves, Adriana V. 2014. \textit{“How Imperialism and the Patriarchy Crushed
\newline 
\hspace*{2cm} Butterfly's Wings."} Accessed 11 05, 2018. http://etd.fcla.edu/CF/CFH0004716/ \\
\hspace*{2cm}Nieves\_Adriana\_V\_201412\_BM.pdf. \\

\newline 2018. \textit{North Korea 'seeks closer South Korea ties, reunification'.} March 6. \\
\hspace*{2cm} Accessed November 12, 2018. https://www.aljazeera.com/news/2018/03/north- 
\hspace*{2cm}korea-seeks-closer-south-korea-ties-reunification-180306062814201.html.\\

n.d. \textit{Royal Anthem.} Accessed November 16, 2018. https://www.canada.ca/en/canadian- 
\hspace*{2cm}heritage/services/royal-symbols-titles/royal-anthem.html#a2.\\

Thoreau, Henry David. 1980. \textit{A Week on the Concord and Merrimack \hspace*{2cm} Rivers.} Princeton, N.J.: Princeton University Press.
\newline \hspace*{2cm} —. 1967. \textit{Walden; or, Life in the Woods.} Paris: Aubier: Editions \\ \hspace*{2cm} Montaigne.
 \\

2018. \textit{Unification 'no matter what': South Koreans on a shared future.} \\ \hspace*{2cm} March 7. Accessed November 16, 2018. https://www.aljazeera.com/indepth/features/ \\
\hspace*{2cm}unification-matter-south-koreans-shared-future-180405131532251.html.
\end{document}
